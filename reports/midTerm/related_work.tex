\section{Related Works}

Asuncion {\it et. al.}~\cite{Asuncion2009smoothing} compare several
approximation algorithms for Latent Dirichlet Allocation (LDA), including
variants of variational Bayes, ML estimation, maximum a posterior (MAP), and
collapsed Gibbs sampling. They report that LDA is more sensitive to the
hyperparameters than the approximation algorithms. However, finding optimal
hyperparameters are generally expensive, and they showed that iterative method
(e.g.~\cite{Minka00}) is not always optimal. Due to the nature of LDA, the
study does not yield performance comparison of classification task. Jiang {\it
et. al.}~\cite{medlda_MCMC12} report that sampling-based approximations
significant out-perform variational methods on classification tasks using
max-entropy discrimination LDA (MedLDA), a supervised variant of LDA.
Both~\cite{Asuncion2009smoothing, medlda_MCMC12} use only text corpus, while
our study include text, ads behavior, genomics data, acceleration data, among
others (sec.~\ref{sec:experiments}). 

Mukherjee {\it et. al.}~\cite{Mukherjee08} provide a theoretical comparison of
two variational Bayes methods for LDA based on mean-field approximation. Here
we consider Bayesian logistic regression which does not lend to mean field
factorization and thus require potentially more challenging variational
methods. Another comparative study over LDA model is by Wallach et
al.~\cite{WallachMM09} where they study diferent priors and the 
corressponding MAP estimates for LDA. Though
they mostly concern them selves with LDA and priors they do enrich the
literature with their valuable analysis and provide some insights regarding the
choice of priors and sampling techniques. 

Holmes et. al~\cite{Holmes} study MCMC estimation techniques over Bayesian logistic
regression in great detail. They build up the bayesian framework for logistic
regression from logit function to all the way upto Kolmogorov-Smirnov prior over
the regression parameter. Since the resultant posterior does not have a nice
closed form,  they provide  rejection based sampling over set of truncated
normals for the parameter updates. The experiments are performed over resonably
diverse datasets, though the only problem is the low dimension of their
datastes. All of their datasets have dimension less than ten which is way
smaller than a real world dataset. We in our experiments have tried to maintain
a diverse set of datasets which deal in very high deimension space. The
compartive analysis changes in the higher dimension since the sampling
strategies tend to be slower and less accurate as one increases the dimension of
the data. 

As one can see there has not been any substantial works done in the area of
statistics or machine learning for a comparaitive study of various approximate
estimation techniques inclding sampling, variational and Laplace techniques, for
graphical models. Besides we also perfomr a comparitive study of these
techniques over twi distinct set of graph objectives: 1) MLE, and 2)MAP. It
would be a worthwhile effort which has a potential to provide working insights
into the world of these complex estimation techniques.

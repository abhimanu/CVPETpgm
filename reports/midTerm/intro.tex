\section{Introduction}

Graphical models have been extensively used in recent decades in the
broader research area of machine learning: from linguistics to network discovery and
structure prediction in parsers to online social media~\cite{Koller+Friedman:09}
Graphical models provide a convenient way of  representing complex structures.
They are a great tool to represent data which has an intuitive generative
or a causal story~\cite{GettingStarted}. In most of these cases we are
estimating the parameters of the probabilistic model where some of the member
nodes of the graphs are not observed or hidden. Models like these are also very
useful in the cases when one has a causal story where there are hidden states of
the in the causal chain for e.g. hidden markov model~\cite{Baum1967}. Graphical
modelling is one of the most effective way to estimate models that involve
transitions over a state space e.g. temporal modelling
\cite{Arnold:2007:TCM:1281192.1281203}. 

There have been various methods
proposed for estimating parameters of such a model e.g. sampling
techniques such as gibbs, metropolis-hasting
etc.~\cite{Robert:2005:MCS:1051451}, variational inference 
techniques~\cite{citeulike:6420690}, Laplace
approximation~\cite{Azevedo-Filho:1994} etc. Each estimation procedure comes
with its own bag of advantages and disadvantages. It is generaly believed that
sampling stratagies are generally more accurate than variational or Laplace,
but the better estimation comes with a price of slower computation. It is also
observed that many sampling strategies suffer from poor mixing of chains for
high dimensional data~\cite{ShenACOSB10}. Laplace approximation tries to
approximate a desired distribution via gaussian. Though the results are poorer
compared to other two in this case, it is believed to be faster than either of
them in most of the cases, but there have been no comparative analyses done for
this technique with other approximation approaches. 

Graphical models also have
variations in terms of the objective one wants to optimize. We can optimize
the likelihood (MLE) of the model directly or a sparse approximation of the
likelihood~\cite{Banerjee:2008}. We can also put a prior on a set of parameters
and obtain a model that maximizes the posterior of the variables using various
optimization techniques~\cite{abs-0710-0013}. We use priors for the parameters
mostly due to two reasons: 1) we have a prior belief about the parameters, or 2)
we want a sparse estimation of the model~\cite{Yoshida:2010}. There are various
quirks and tricks with lots of unsaid rules that go in all these optimization
techniques. Besides one can estimate a graphical model by choosing any of these
two objectives and maximize it using any of the estimation techniques mentioned
above. So we have a whole gamut of possibilities about which not much is known. 

Our goal in this project is to clear the uncertainty about choice of the
objective and the corressponding estimation technique to choose. We aim to
perform an empirical study of various combinations of these startegies and
objectives to be optimized on a suitable graphical model problem. Our choice of
the problem is bayesian logistic regression~\cite{Xu:2008:BLR}. Bayesian
logistic regression is a widely used model and thus a suitable choice for this
particular analysis. We also plan to do comparitive study on other models such
as LDA etc. if time permits.

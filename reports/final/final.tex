\documentclass{article}

\usepackage{nips12submit_e, times}
\usepackage{hyperref}
\usepackage{graphicx}
\usepackage{amsmath}
\usepackage{amsfonts}
\usepackage{amssymb}
\usepackage{bm}

% ======= Dai Wei's preambles. Please don't delete =======
\newcommand{\phib}{\bm{\phi}}
\newcommand{\vecu}[1]{\underline{\mathbf{#1}}}
\newcommand{\wb}{\mathbf{w}}
\newcommand{\xb}{\mathbf{x}}
\newcommand{\defeq}{\stackrel{\Delta}{=}}
\DeclareMathOperator*{\argmax}{arg\,max}
% ======= Dai Wei's preambles. Please don't delete =======

\title{An Empirical Study of Approximate Inference Algorithms on Bayesian
Logistic Regression}

\author{
Wei Dai \\
School of Computer Science \\
Carnegie Mellon University\\
Pittsburgh, PA 15213\\
\texttt{wdai@cs.cmu.edu}\\
\And
Abhimanu Kumar \\
School of Computer Science \\
Carnegie Mellon University\\
Pittsburgh, PA 15213\\
\texttt{abhimank@cs.cmu.edu}\\
\And
Jinliang Wei \\
School of Computer Science \\
Carnegie Mellon University\\
Pittsburgh, PA 15213\\
\texttt{jinlianw@cs.cmu.edu}\\
}
\newcommand{\fix}{\marginpar{FIX}}
\newcommand{\new}{\marginpar{NEW}}
\nipsfinalcopy
\begin{document}
\maketitle

\begin{abstract}
We present a comparative study of various approximation algorithms present in
the literature for probabilistic graphical model inference and estimation. There
is a dilemma in considerable number of research populace regarding the choice
of the approach for estimating graphical model parameters. There is no hard and
fast rule as to when to use what. We attempt to present here an empirical study of
various approximation algorithm, especially the big two: sampling based
and variational based, in use for graphical model learning. Though our model for
learning is restricted to Bayesian logistuic regression it does present a
consoderable challege since there is non-conjugacy due to presence of sigmoid
function for the label prediction. We presnt clever ways to work-around such
situation in variational as well as sampling methods. We analyze the algorithms
on a diverse set of datasets as well as present run time results for speed
comparison.
\end{abstract}

\setcounter{page}{1}
\pagenumbering{arabic}
% \vspace{-20pt}
% \section{Project Idea}
% \vspace{-5pt}

\section{Introduction}

Graphical models have been extensively used in recent decades in the broader
research area of machine learning: from linguistics to network discovery and
structure prediction in parsers to online social
media~\cite{Koller+Friedman:09} Graphical models provide a convenient way of
representing complex structures.  They are a great tool to represent data
which has an intuitive generative or a causal story~\cite{GettingStarted}. In
most of these cases we are estimating the parameters of the probabilistic
model where some of the member nodes of the graphs are not observed or hidden.
Models like these are also very useful in the cases when one has a causal
story where there are hidden states of the in the causal chain for e.g. hidden
markov model~\cite{Baum1967}. Graphical modelling is one of the most effective
way to estimate models that involve transitions over a state space e.g.
temporal modelling \cite{Arnold:2007:TCM:1281192.1281203}. 

There have been various methods proposed for estimating parameters of such a
model e.g. sampling techniques such as Gibbs, Metropolis-Hasting
etc.~\cite{Robert:2005:MCS:1051451}, variational inference
techniques~\cite{citeulike:6420690}, Laplace
approximation~\cite{Azevedo-Filho:1994} etc. Each estimation procedure comes
with its own advantages and disadvantages. For instance, it is generally
believed that sampling stratagies are more accurate than variational or
Laplace, but the better estimation comes with a price of slower computation.
It is also observed that many sampling strategies suffer from poor mixing of
chains for high dimensional data~\cite{ShenACOSB10}. Laplace approximation
tries to approximate potentially complex posterior distribution with
multivariate Gaussian. Although the results are poorer than variational and
sampling, Laplace approximation is often computationally faster in most cases.
Despite evidences of these trade-offs~\cite{Asuncion2009smoothing,
medlda_MCMC12}, to the best of our knowledge, there is no comprehensive
empirical studies of these approximation techniques on classification tasks. 

Graphical models also have variations in terms of the objective one wants to
optimize. We can optimize the likelihood (MLE) of the model directly or a
sparse approximation of the likelihood~\cite{Banerjee:2008}. We can also put a
prior on a set of parameters and obtain a model that maximizes the posterior
of the variables using various optimization techniques~\cite{abs-0710-0013}.
We use priors for the parameters mostly due to two reasons: 1) we have a prior
belief about the parameters, or 2) we want a sparse estimation of the
model~\cite{Yoshida:2010}. There are various quirks and tricks with lots of
unsaid rules that go in all these optimization techniques. Besides one can
estimate a graphical model by choosing any of these two objectives and
maximize it using any of the estimation techniques mentioned above. So we have
a whole gamut of possibilities about which not much is known. 

Our goal in this project is to clear the uncertainty about choice of the
objective and the corressponding estimation technique to choose. We aim to
perform an empirical study of various combinations of these startegies and
objectives to be optimized on a suitable graphical model problem. Our choice
of the problem is bayesian logistic regression~\cite{Xu:2008:BLR}. Bayesian
logistic regression is a widely used model and thus a suitable choice for this
particular analysis. We also plan to do comparitive study on other models such
as LDA etc. if time permits.



Asuncion {\it et. al.}~\cite{Asuncion2009smoothing} compare several
approximate algorithms for Latent Dirichlet Allocation (LDA), including
variants of variational Bayes, ML estimation, maximum a posterior (MAP), and
collapsed Gibbs sampling. They report that LDA is more sensitive to the
hyperparameters than the approximation algorithms. However, finding optimal
hyperparameters are generally expensive, and they showed that iterative method
(e.g.~\cite{Minka00}) is not always optimal. Due to the nature of LDA, the
study does not yield performance comparison of classification task. Jiang {\it
et. al.}~\cite{medlda_MCMC12} report that sampling-based approximations
significant out-perform variational methods on classification tasks using
max-entropy discrimination LDA (MedLDA), a supervised variant of LDA.
Both~\cite{Asuncion2009smoothing, medlda_MCMC12} use only text corpus, while
our study include text, ads behavior, acceleration data, among others (sec.
experiment.) TODO 

Mukherjee {\it et. al.}~\cite{Mukherjee08} provide a theoretical comparison of
two variational Bayes methods for LDA based on mean-field approximation. Here
we consider Bayesian logistic regression which does not lend to mean field
factorization and thus require potentially more challenging variational
methods. 


Given data set $\{\phib_n, t_n\}_{n=1}^N$ where $\phib_n$ are the feature
vectors and $t_n\in \{0,1\}$ are the labels, we can write the likelihood
function for logistic regression as $p(\bm{t}|\bm{w}) = \prod_{n=1}^N
y_n^{t_n} (1-y_n)^{1-t_n}$ where $\bm{t} = (t_1,...,t_N)^T$ and
$y_n=p(\mathcal{C}_1|\phib_n) = \sigma(\bm{w}^T \phib_n)$ and $\sigma(s) =
\frac{1}{1+e^{-s}}$. Using Bayes rule, the posterior distribution over
$\bm{w}$ is $p(\bm{w}|\bm{t}) = \frac{p(\bm{w}) p(\bm{t}|\bm{w})}{p(\bm{t})}$
where $p(\bm{t}) = \int p(\bm{w})p(\bm{t}|\bm{w}) d\bm{w}$ involves logistic
sigmoid functions and is intractable. In the sequel we briefly describe three
approximation schemes (Laplace approximation, variational methods, and Gibbs
sampling), and point estimations (MLE, MAP) together with the associated
prediction rules.

\subsection{Laplace Approximation and associated MAP}

Laplace approximation approximate the posterior $p(\bm{w}|\bm{t})$ with a
multivariate Guassian $\mathcal{N}(\bm{w}; \bm{w}_{MAP}, \bm{S}_N)$ where
$\bm{w}_{MAP}$ is the maximum {\it a posteriori} and thus a mode of the
posterior and $\bm{S}_N^{-1} = -\nabla^2_{\bm{w}} \ln
p(\bm{w}|\bm{t})|_{\bm{w} = \bm{w}_{MAP}}$ is the Hessian at $\bm{w}_{MAP}$.
Since we are approximating the posterior with Gaussian, it is convenient to
use conjugate prior $p(\bm{w}) = \mathcal{N}(\bm{w};\bm{m}_0,\bm{S}_0)$. Thus
we have $\ln p(\bm{w}|\bm{t}) = -\frac{1}{2}(\bm{w}-\bm{m}_0)^T
\bm{S}_0^{-1}(\bm{w}-\bm{m}_0) + \sum_{n=1}^N\{t_n \ln y_n +(1-y_n) \ln
(1-y_n)\} + const$

Under the Laplace approximation and the conjugated Gaussian prior, $\bm{w}_{MAP}$
can be efficiently obtained by gradient descent. To encourage small
$||w||^2_2$, let $\bm{m}_0 = \bm{0}$, and $\bm{S}_0 = \sigma^2 \bm{I}$, we
have the following gradient descent rule: 

\begin{equation}
\bm{w}_t \leftarrow \bm{w}_{t-1} + \eta\left( \sum_{n=1}^N (t_n - y_{n,(t-1)}) \phib_n -
\frac{1}{\sigma^2}\bm{w}_{t-1} \right)
\end{equation}

where $\eta$ is the learning rate constant. We can also get

\begin{equation}
\bm{S}_N^{-1} = -\nabla^2_{\bm{w}} \ln p(\bm{w}|\bm{t})
= \bm{S}_0^{-1} + \sum_{n=1}^N y_n(1-y_n) \phib_n \phib_n^T
\end{equation}

The predictive distribution for Laplace-approximated posterior is not close
form, but by approximating the logistic sigmoid function with probit function,
we recover $\bm{w}_{MAP}$ as the decision boundary.


\section{Experiments}
\label{sec:experiments}

\subsection{Data Sets}


We use 5 \href{http://archive.ics.uci.edu/ml/datasets.html}{UCI classification
datasets}:, described as follows:

\begin{tabular}{| c | c |  c |}
\hline
Dataset & \# instances & \# features \\
\hline
Farms Ads dataset & 4,143 & 54,877 \\
\hline
Amazon Commerce reviews dataset & 1,500 & 10,000 \\
\hline
p53 Mutants dataset & 16,772 & 5,409 \\
\hline
Human Activity Recognition using Smartphones dataset & 10,299 & 561\\
\hline
URL Reputation dataset\footnotemark[1] & 2,396,130 (16000) & 3,231,961 (74113) \\
\hline
\end{tabular}

\footnotetext[1]{The dataset is too large so we used only one day of data out of totally 121 days in the dataset. See number in parentheses for specifications.}

\subsection{Results}

\subsubsection{MCMC based estimation}
Our MCMC sampling strategy that we described in section~\ref{sec:MCMCmethod}
converges. Figure~\ref{fig:MCMCconverge} is a plot of the iterations of the Markov 
chain for estimation on a subset of Farms Ads dataset. This is for MLE
estimation without any priors. We also achieve a training  accuracy of 8.57\%
and a test accuracy of 9.12\% over this subset of datset.  

\begin{figure}![htb]
\includegraphics[width=1\textwidth]{samplingConvergence.png}
\caption{Markov chain convergence on Farms ads data. We take a sub-sample of
the dataset: 2000 rows and 100 columns}
\label{fig:MCMCconverge}
\end{figure}


\section{Conclusion an Future Work}
As we can see Kolmogorov Smirnov based sampling based sampling performs
considerable well compared all other methods in terms of raw accuracy of
prediction. Though it is slower in speed compared to other variational methods
such as Laplace and Jordan's method. Sampling is harder to tune due to
asymptotic natur eof the approach. It also suffers from bad-mixing if the
feature space os very large as well as when there is high correlaton between
successive particles sampled. This slow mixing can be clearly observed the
uniform distribution based sampling approach for Bayesian logistic regression. 

The three algorithms used in this paper almost surely cover the
span of various styles of approximation algorithms used in all graphical model
estimations. There are other approximation algorithms such as loopy belief
propagation for general graphs but these are variants of the variational mehods
covered in this work~\cite{Heskes02}. We covered broad 

Sampling performs
comparatively better than the other approximate inference schemes on all datasets

    But sampling has many parameters to be tuned

    Variational schemes are sensitive to step size of the ascent while optimizing

    MCMC based sampling is slower than most variational inference (except delta method).
    But can be fast when there is good mixing in the chains



\bibliographystyle{plain}
\bibliography{reference}

\end{document}

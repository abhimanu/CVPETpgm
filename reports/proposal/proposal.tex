\documentclass[a4paper, 10pt]{article}

\usepackage{nips12submit_e, times}
\usepackage{hyperref}

\title{An Empirical Study of Approximation Inference Algorithms on Bayesian Logistic Regression}

\author{
Wei Dai \\
School of Computer Science \\
Carnegie Mellon University\\
Pittsburgh, PA 15213\\
\texttt{wdai@cs.cmu.edu}\\
\And
Abhimanu Kumar \\
School of Computer Science \\
Carnegie Mellon University\\
Pittsburgh, PA 15213\\
\texttt{abhimank@cs.cmu.edu}\\
\And
Jinliang Wei \\
School of Computer Science \\
Carnegie Mellon University\\
Pittsburgh, PA 15213\\
\texttt{jinlianw@cs.cmu.edu}\\
}
\newcommand{\fix}{\marginpar{FIX}}
\newcommand{\new}{\marginpar{NEW}}
\nipsfinalcopy
\begin{document}
\maketitle

\setcounter{page}{1}
\pagenumbering{arabic}
\vspace{-20pt}
\section{Project Idea}
\vspace{-5pt}
Latent variable models have gained significant popularity due to their ability to model data that potentially arise due to latent causes. These models are also helpful in case of missing data as these missing entries can be modelled via hidden variables in graphical models. The parameter estimation for this class of models are in general intractable, and numerous approximation algorithms are widely employed. These approaches fall broadly into two categories: 1) variational inference, and 2) sampling-based. While it is generally believed that sampling produces better approximation than variational inference, albeit at a higher computational cost, to the best of our knowledge there is no comprehensive empirical study that compares these approximation inference schemes.

We plan to carry out an empirical comparative study of these approximation algorithms on Bayesian logistic regression, a well-studied minimal model with only one latent variable: the regression coefficients. The approximation algorithms we will use are:
\vspace{-5pt}
\begin{enumerate}
  \item Variational inference \cite{RePEc:bes:jnlasa:v:105:i:489:y:2010:p:324-335}:
    \begin{enumerate}
      \item A (near) close-form variational bound \cite{Jaakkola96avariational}
      \item Laplace variational inference and delta method variational inference \cite{2012arXiv1209.4360W}
    \end{enumerate}
    \item Sampling-based: MCMC using Gibbs sampling
\end{enumerate}
\vspace{-5pt}
We would apply the above approximation algorithms to the following estimation problems for a comparative study:
\vspace{-5pt}
\begin{enumerate}
  \item Maximum a posteriori (MAP) from variational and sampling algorithms
  \item Maximum likelihood estimation (MLE)
  \item Laplace approximation
\end{enumerate}
\vspace{-5pt}
\section {Software, Datasets and Midterm Milstone}
\vspace{-5pt}
We plan to write all the code in matlab or C++ so that we have language agnostic comparison results. We will use 5 \href{http://archive.ics.uci.edu/ml/datasets.html}{UCI classification datasets}: 1) Farms Ads dataset (4,143 instances, 54,877 text features), 2) Amazon Commerce reviews dataset (1,500 instances, 10,000 real features) 2) p53 Mutants dataset (16,772 instances, 5,409 real attributes) 4) Human Activity Recognition using Smartphones dataset (10,299 instances, 561 real features) 5) URL Reputation dataset (2,396,130 instances, 3,231,961 real features).

We plan to finish MCMC sampling for all the three estimation by midterm.

See reference for papers to read.
\bibliographystyle{plain}
\bibliography{reference}

\end{document}